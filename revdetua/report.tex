\documentclass[mirror, portugues]{revdetua}

% Valid options are:
%   portugues --------- main language is Portuguese
%   final ------------- final version (default)
%   times ------------- use times (postscript) fonts for text
%   mirror ------------ prints a mirror image of the paper (with dvips)
%   visiblelabels ----- \SL, \SN, \SP, \EL, \EN, etc. defined
%   invisiblelabels --- \SL, \SN, \SP, \EL, \EN, etc. not defined (default)
%
% Note: the final version should use the times fonts
% Note: the really final version should also use the mirror option

\usepackage[portuguese]{babel}
\usepackage[utf8]{inputenc}
\usepackage{amsmath} 
\usepackage{comment}
\usepackage{algorithm}
\usepackage{algpseudocode}
\floatname{algorithm}{Algoritmo}
\usepackage{graphicx}
\usepackage[justification=centering]{caption}
\usepackage{float}
\usepackage{booktabs}
%-------------------------------------
% compiling:
% Recipe: xelatex
% Recipe: pdflatex -> bibtex -> pdflatex -> pdflatex
% Recipe: xelatex
%
% notas:
% rever se algoritmos e imagens estão onde devem
%-------------------------------------
\begin{document}

\Header{03}{3}{Janeiro}{2025}{1}

\title{TITULO DO TRABALHO}
\author{Hugo Veríssimo - 124348 - hugoverissimo@ua.pt}
\maketitle

\begin{abstract}
abstrato em ingles
\end{abstract}

\begin{resumo}
abstrato em pt resumo
\end{resumo}


\section{Introdução}

ns q big data, problemas de contagem, varios metodos, importancia. 

um exemplo é nos lviros, contar palavras pq sao mt extensos e o guardar memoria tem um custo

\section{Metodologia da Análise ?}

foram tirados livros em 3 lignaus do piqnoquio, do site tal 

foi usado o spacy para meter tudo lower, lemm, stpowords, pontuacao, etc

...

\section{contagem 1}

O primeiro algoritmo é a contagem toda \cite{EX69}.

\begin{algorithm}[H]
\raggedright
\textbf{Entrada:}

- lista de arestas e respetivos pesos (\textit{edges})

- número de vértices (\textit{n\_nodes})

- número de soluções a gerar (\textit{solutions})\\
\textbf{Saída:} subconjuntos \textit{S} e \textit{T}, peso do corte (\textit{weight}) \\
\hrule 
\caption{Corte Aleatório}
\begin{algorithmic}[1]
    \State \texttt{best\_solution} $\gets$ \texttt{None}
    \State \texttt{weight} $\gets$ 0
    \State \texttt{seen\_solutions} $\gets$ empty set
    \For{$i \gets 1$ \textbf{to} \texttt{solutions}}
        \State \texttt{partition} $\gets$ random partition of the nodes
        \If {\texttt{length}(\texttt{seen\_solutions}) $=$ $2^{\texttt{n\_nodes}}$}
            \State \textbf{break}
        \EndIf
        \State \texttt{partition\_hash} $\gets$ hash the partition
        \If {\texttt{partition\_hash} $\in$ \texttt{seen\_solutions}}
            \State \textbf{continue}
        \EndIf
        \State Add \texttt{partition\_hash} to \texttt{seen\_solutions}
        \State \texttt{new\_cut\_weight} $\gets$ compute the cut weight
        \If {\texttt{new\_cut\_weight} $>$ \texttt{weight}}
            \State \texttt{weight} $\gets$ \texttt{new\_cut\_weight}
            \State \texttt{best\_solution} $\gets$ copy of \texttt{partition}
        \EndIf
    \EndFor
    \State \texttt{S} $\gets$ set of nodes assigned to $0$ in \texttt{best\_solution}
    \State \texttt{T} $\gets$ set of nodes assigned to $1$ in \texttt{best\_solution}
    \Return \texttt{S}, \texttt{T}, \texttt{weight}
\end{algorithmic}
\end{algorithm}
    

%\begin{figure}[H]
%    \centering
%    \includegraphics[width=0.45\textwidth]{../assets/ops_Random Sol.png}
%    \caption{Número de operações básicas efetuadas pelo algoritmo de Corte Aleatório em função do número de arestas do grafo, para diferentes valores de \texttt{solutions} (MS).}
%    \label{fig:random_ops}
%\end{figure}

Quanto ao número de soluções testadas, a partir da Fig. 

%\ref{fig:sols_randomrandom}

\section{contagem 2}

lalalla


\section{contagem 3}

lalallala

\section{resultados}

\begin{table}[H]
\centering
\caption{CAPTION CAPTION CAPTION}
\label{table:numops}
\begin{tabular}{ll}
\toprule
\textbf{Algoritmo} & \textbf{Complexidade} \\
\midrule
a & $O(m)$ \\
b & $O(m)$ \\
c & $O(m^2 \times n)$ \\
\bottomrule
\end{tabular}
\end{table}

ffyf

\section{Conclusão}

conclusaoooo

\bibliography{refs}

\end{document}
